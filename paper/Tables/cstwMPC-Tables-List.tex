\begin{sidewaystable}
\caption{Empirical Estimates of the Marginal Propensity to Consume (MPC) out of Transitory Income}
\label{table:mpcLit}
\begin{minipage}{\textwidth}
\input \import{/Volumes/Data/Work/BufferStock/Macro/cstwMPC/Latest/}{Tables/mpcLit.tex}
\tablenotessize{Notes: $^{\star}$: The horizon for which consumption response is calculated is 3 months or 1 year.
The papers which estimate consumption response over the horizon of 3 months typically suggest that the response
thereafter is only modest, so that the implied cumulative MPC over the full year is not much higher than over the first three months.
$^{\ddagger}$:~elasticity.

\citet{brodaParker:stimulus2008} report the five-month cumulative MPC of 0.0836--0.1724 for the consumption goods in their dataset. However, the Homescan/NCP data they use only covers a subset of total PCE, in particular grocery and items bought in supercenters and warehouse clubs.
We do not include the studies of the 2001 tax rebates, because our interpretation of that event is that it reflected a permanent tax cut that was not perceived by many households until the tax rebate checks were received.  While several studies have examined this episode, e.g., \citet{shapiroSlemrod:AER03}, \citet{jpsTax}, \citet{aslCredit} and \citet{misraSurico:heteroResponses}, in the absence of evidence about the extent to which the rebates were perceived as news about a permanent versus a transitory tax cut, any value of the MPC between zero and one could be justified as a plausible interpretation of the implication of a reasonable version of economic theory (that accounts for delays in perception of the kind that undoubtedly occur).}
\end{minipage}
\end{sidewaystable}
\begin{table}
\caption{Parameter Values and Steady State}
\label{table:ParamsAll}
\begin{minipage}{\textwidth}
\input \import{/Volumes/Data/Work/BufferStock/Macro/cstwMPC/Latest/Tables/}{ParamsAll}
\tablenotessize{Notes: The models are calibrated at the quarterly frequency, and the steady state values are calculated on a quarterly basis.}
\end{minipage}
\end{table}
\begin{sidewaystable}
\caption{Average (Aggregate) Marginal Propensity to Consume in Annual Terms}
\label{table:MPCall}
\begin{minipage}{\textwidth}
\input \import{/Volumes/Data/Work/BufferStock/Macro/cstwMPC/Latest/Tables/}{MPCall}
\tablenotessize{Notes: Annual MPC is calculated by $1-(1-$\text{quarterly MPC}$)^{4}$. ${}^\ddagger$: Discount factors are uniformly distributed over the interval $[\grave{\Discount}-\nabla,\grave{\Discount}+\nabla]$.
%$^{\ddagger}:$ $\grave{\Discount}=
%\input ../Code/Mathematica/Results/Beta.tex
%$.
%$^{\star}:$ $(\grave{\Discount}, \nabla)=(
%\input ../Code/Mathematica/Results/Betamiddle.tex
%,
%\input ../Code/Mathematica/Results/nabla.tex
%)$, which implies $\{\grave{\Discount}-\nabla, \grave{\Discount}+\nabla\}=\{
%\input ../Code/Mathematica/Results/BetaLow.tex
%,
%\input ../Code/Mathematica/Results/BetaHigh.tex
%\}$.
}
\end{minipage}
\end{sidewaystable}
\begin{table}
\caption{Proportion of Wealth Held by Percentile (in Percent)}
\label{table:LiqFinPlsRetAssetDist}
\begin{minipage}{\textwidth}
\input \import{/Volumes/Data/Work/BufferStock/Macro/cstwMPC/Latest/Tables/}{LiqFinPlsRetAssetDist_2}
\tablenotessize{Notes: The data source is the 2004 Survey of Consumer Finances.}
\end{minipage}
\end{table}
